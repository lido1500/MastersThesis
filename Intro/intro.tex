\section*{Problématique}

Tout comme l'univers, l'Internet est en expansion continue. Selon l'agence
comScore, plus de 170 millions d'États-Uniens ont visionné de la vidéo par
l'intermédiaire du réseau Internet durant le mois de février 2011
\citep{ComScore2011}. De plus, \citet{ComScore2011} ajoute que durant ce mois,
ce même public a effectué plus de 5 milliards de visionnements de séquences
vidéos. Pour sa part, l'agence \citet{Juniper2010} affirme que la vidéophonie
atteindra 29 millions d'utilisateurs d'ici 2015.

Devant ces chiffres, on ne peut que constater la croissance fulgurante de la
consommation de la vidéo sur des réseaux peu fiables, tels Internet ou, dans des
conditions \textit{temps réel}, comme c'est le cas avec la vidéophonie. Cette
tendance de consommation se manifeste aussi dans les réseaux mobiles. Dans son
rapport, \citet{Nielsen2011} indique qu'en novembre 2010, aux États-Unis, 45\%
des téléphones achetés dans les six derniers mois étaient des téléphones
intelligents et que, pour ce même mois, 31\% des téléphones sur le marché
états-unien étaient des téléphones intelligents, donc capables d'accéder à
Internet et de consulter des flux vidéos sur des réseaux mobiles. L'augmentation
du nombre d'utilisateurs et leur soif de contenu de haute qualité accaparent les
ressources limitées des opérateurs de réseaux mobiles.

Dans ces conditions, il est crucial d'optimiser l'usage du canal de
transmission. Cette optimisation ne vient pas seulement de l'amélioration des
taux de compression, quoique la norme H.264 permette d'obtenir des taux de
compression considérablement supérieurs aux normes antérieures, mais vient aussi
de l'amélioration des mécanismes de résilience aux erreurs. L'amélioration de la
résilience aux erreurs diminue la charge sur un réseau en réduisant le nombre de
retransmissions, lorsque celles-ci sont envisageables. La vidéophonie sur des
appareils mobiles est une des conditions où la retransmission n'est pas
envisageable. De plus, la charge accrue sur les réseaux mobiles peut, elle
aussi, limiter l'efficacité de la retransmission. La retransmission peut aussi
mener à des délais qui ne sont pas acceptables pour l'application vidéo
envisagée. La vidéoconférence en est un parfait exemple.

La particularité du réseau mobile est qu'il est particulièrement vulnérable à
l'erreur. Il est souvent considéré comme le maillon faible dans la chaine de
transmission de flux vidéos ou de la vidéophonie. Le parcours des données vidéos
du réseau Internet jusqu'au réseau de l'opérateur est fait, généralement, sur
des réseaux câblés. En ce qui concerne ces pertes encourues lors de ce
transport, il n'y a rien à faire, car les données ne se rendent pas à
destination. Une fois rendues chez l'opérateur, les données se rendent à la
station de base \ang{base transceiver station} appropriée à la cellule de
l'utilisateur et sont transmises sur le lien sans fil. Cette transmission est
très sensible et, comme mentionnée, selon les conditions du réseau mobile, la
retransmission sur ce lien n'est pas toujours possible ni envisageable.
Lorsqu'il y a corruption sur le lien sans fil, les données se rendent à
l'appareil mobile, mais un certain nombre de bits sont altérés par
l'interférence. Dans ce cas, l'ensemble du paquet est rejeté, ceci représente un
gaspillage important d'information utile, car les bits non altérés du paquet
sont eux aussi supprimés. N'ayant pas la possibilité de retransmission, ne
serait-il pas mieux d'exploiter l'information du paquet, même si une partie de
celle-ci est corrompue?

Pour résoudre ce problème, il faut changer de paradigme. La puissance de
traitement n'est plus uniquement disponible dans les serveurs colossaux des
opérateurs, mais aussi répartie à travers les appareils mobiles sur le réseau.
Ceux-ci sont de plus en plus puissants, comme en témoignent les spécifications
techniques du iPhone 4, avec son processeur d'un gigahertz et sa mémoire vive de
512 mégaoctets.

Il est possible de tirer profit de cette nouvelle puissance computationnelle
pour reconstruire les données corrompues lors du transport sur un réseau peu
fiable. Contrairement à l'encodeur qui cherche à extraire la redondance d'une
séquence, ce nouveau type d'algorithme cherche à utiliser la redondance des
images ou des portions d'images décodées pour reconstruire ce qui a été
endommagé, et ce, sans l'ajout, apriori, de données redondantes de la part de
l'encodeur.

\section*{Motivations}
Un algorithme capable d'identifier et de dissimuler la détérioration visuelle
issue des erreurs de transport pourrait transformer l'écosystème mobile. Non
seulement serait-il capable de réduire la charge sur les réseaux mobiles, mais
il pourrait aussi augmenter la qualité visuelle perçue par l'utilisateur dans
des contextes où la retransmission est impraticable, comme c'est le cas en
vidéophonie.

De plus, cet algorithme pourrait remplacer en partie certains mécanismes de
résilience à l'erreur qui requièrent une quantité considérable de bande
passante, tels la retransmission ou l'ajout de trames redondantes. Sans pour
autant éliminer ces méthodes, les opérateurs, sachant que le décodeur est plus
résilient à l'erreur, pourraient réduire considérablement la quantité de bits
alloués à la résilience lors du transport.

Notons aussi qu'une solution qui ne requiert des altérations qu'au
décodeur fait en sorte qu'elle est compatible avec la quantité incommensurable
de contenu vidéo déjà encodé, et évite d'avoir à tout réencoder pour
assurer la compatibilité avec un nouveau mécanisme.

\section*{Objectifs}
Notre effort de recherche vise à déterminer, tout d'abord, s'il est possible de
décoder avec succès (sans plantage) une séquence corrompue et, si oui, quelle en
est la probabilité. Par la suite, lorsque ce genre de décodage est réussi, nous
cherchons à identifier les caractéristiques de la détérioration visuelle
engendrée par le train de bits corrompu traité par le décodeur. Puis, nous
évaluerons les possibilités de dissimuler cette détérioration visuelle.

Notre objectif ultime est de concevoir et de développer un algorithme capable de
détecter et de dissimuler les dégradations visuelles. Cet algorithme doit
satisfaire aux exigences logicielles suivantes:
\begin{itemize}
\item L'algorithme doit identifier la détérioration visuelle importante
résultant du décodage d'une trame corrompue.
\item L'algorithme doit dissimuler la détérioration visuelle importante
résultant du décodage d'une trame corrompue, dans le but d'améliorer la qualité
visuelle de la trame.
\item L'algorithme doit interagir uniquement avec le décodeur, et ce seulement
dans le domaine des pixels, soit à la suite du décodage de la trame.
\end{itemize}

\section*{Organisation du mémoire}
Dans cet ouvrage, les informations sont logiquement regroupées en chapitres. La
chronologie des chapitres permet de présenter les notions élémentaires en
premier, suivies de l'état de l'art, de la solution proposée et des résultats
des simulations servant à mesurer l'efficacité de notre solution.

Les deux premiers chapitres présentent les notions de base requises pour une
bonne compréhension de la norme H.264 et des mécanismes pour en effectuer le
transport sur des réseaux peu fiables. Un lecteur avec ces connaissances peut
les esquiver et poursuivre sa lecture au chapitre trois.

Le troisième chapitre fait état des conséquences du décodage de séquences H.264
corrompues. On y présente les répercussions de la corruption d'un encodage
entropique, les différents types de détérioration visuelle ainsi que les sources
de propagation de cette détérioration. Ce chapitre présente un contenu rare, de
par le fait que, dans la majorité des cas, les paquets erronés sont rejetés et
non pas décodés.

Au quatrième chapitre, nous passons en revue la littérature portant sur la
détection de la détérioration visuelle dans les images et les séquences vidéos.
Pour ce faire, nous présentons aussi l'évolution des efforts de recherche dans
les domaines connexes, tels l'analyse syntaxique de l'encodage vidéo et le
\textit{Joint source channel decoding}.

Le cinquième chapitre présente la solution proposée. Tout d'abord, nous
présentons une mesure de la détérioration visuelle, basée sur les effets de
blocs compensés par le mouvement. Par la suite, nous présentons deux approches
sélectives capables d'utiliser notre mesure pour guider les choix de
dissimulation.

Au sixième chapitre, nous présentons les résultats de nos simulations dans le
but de mesurer l'efficacité des approches proposées. Finalement, au dernier
chapitre, nous concluons cet ouvrage.