Dans ce mémoire, nous avons présenté, tout d'abord, les notions de base sur la
norme H.264 et son transport. Par la suite, nous avons étudié la détérioration
visuelle issue du décodage de trames corrompues. Une revue de la littérature a
aussi été effectuée, portant non seulement sur les approches de détection de la
détérioration visuelle, mais aussi sur l'analyse syntaxique de vidéos encodés et
le \textit{Joint source channel decoding}.

Par ailleurs, nous avons présenté une nouvelle mesure permettant d'identifier
les effets de bloc propres à la détérioration visuelle. Nos analyses des
simulations permettent les conclusions suivantes:
\begin{itemize}
\item Il est possible de décoder des trames corrompues. La probabilité d'un
décodage réussi varie de 20~\% à 70~\% selon les paramètres d'encodage et le
taux d'erreurs subi lors du transport.
\item Les trames endommagées peuvent offrir une dissimulation de meilleure
qualité que le calque de la trame. Pour notre banc d'essai, 44~\% des séquences
démontraient un PSNR supérieur au calque de la trame. Ce pourcentage peut
fluctuer selon la variation \textit{intertrame} et l'importance de la
détérioration visuelle présente dans la trame.
\item La mesure des effets de bloc compensés par le mouvement permet de
départager entre deux candidats de dissimulation, lequel possède le moins de
détérioration visuelle importante. Pour notre banc d'essai, elle a effectué le
bon choix dans 81~\% et 86~\% des cas, selon le FMO, pour l'approche sélective
par trame et, dans 88~\% et 91~\% des cas, selon le FMO, pour celle par bloc.
\end{itemize}

L'algorithme réalisé répond aux objectifs du projet. Comme démontré, il est
capable d'identifier et de dissimuler la détérioration visuelle. Il offre des
gains moyens de 0.72~dB et de 0.65~dB, selon le FMO, pour l'approche sélective
par trame et de 0.86~dB et de 0.69dB, selon le FMO, pour celle par bloc, par
rapport au calquage de la trame. De plus, il n'utilise que l'information
disponible dans le domaine des pixels.

Quoique la solution proposée offre des résultats intéressants, il reste encore
beaucoup de possibilités d'amélioration, comme, par exemple, l'implémentation de
notre solution dans un décodeur H.264 commercial. Il serait aussi intéressant
d'effectuer l'intégration avec d'autres algorithmes de dissimulation, tel le
calque des vecteurs de mouvements. 

De plus, il serait aussi possible de combiner notre approche avec de
l'information provenant des paramètres d'encodage, tels l'ordonnancement des
macroblocs, le nombre de tranches, etc. Ce type d'information pourrait améliorer
l'efficacité du MCB. Ce type d'approche n'a pas été étudié dans cet ouvrage,
car notre objectif était de travailler uniquement dans le domaine des pixels.

Sous leur forme actuelle, les concepts présentés dans cet ouvrage peuvent non
seulement améliorer la dissimulation chez des décodeurs H.264, mais aussi servir
d'outils d'analyse de la détérioration visuelle sans référence. Ce genre
d'outils peuvent être insérés dans des décodeurs numériques ou des appareils
mobiles, comme ceux offerts par les opérateurs comme Rogers, Telus, Bell et
Vidéotron, et faire état de la présence de détérioration visuelle s'il y a lieu.

Tout cela réuni fait en sorte que, quoique notre effort de recherche se termine,
les contributions issues de cet ouvrage ont un potentiel important de
développement et d'évolution dans d'autres projets de recherche ou des
initiatives commerciales.
